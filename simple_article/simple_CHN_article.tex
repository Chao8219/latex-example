\documentclass[a4paper]{article}
\usepackage[utf8]{inputenc}
\usepackage{xeCJK}
\usepackage[nottoc]{tocbibind} % includes reference in toc
\usepackage{doc} % includes \BibTeX logo
\usepackage{metalogo} % includes \XeLaTeX logo
\usepackage{fontspec} % specify fonts
\usepackage[toc]{glossaries} % includes glossaries in toc

\renewcommand{\refname}{参考文献 Reference}
\renewcommand{\contentsname}{目录 Content}

\title{
    中英文混排 Article \\
    \large with English Subtitle
}
\author{作者 Author:Chao}
\date{2019年3月}

% glossaries
\newglossaryentry{Lao She}{
	name={老舍},
    description={舒庆春(1899年2月3日-1966年8月24日),
    字舍予,笔名老舍}
}
\newglossaryentry{Zhang Ruoxu}{
    name={张若虚},
    description={张若虚(约660年-约720年),
    唐代诗人,扬州人}
}
\makeglossaries
\setglossarystyle{altlist} % indent with newline for discription

% set fonts
\setromanfont[
    Path=../fonts/,
    BoldFont=timesbd.ttf,
    ItalicFont=timesi.ttf,
    BoldItalicFont=timesbi.ttf
]{times.ttf}
\setCJKmainfont[
    Path=../fonts/
]{simsun.ttc}

\begin{document}
\maketitle
\tableofcontents
\newpage
\section{唐诗 Tang Poetry}
    \subsection{简体中文版 Simplfied Chinese}
        \begin{center}
            春江潮水连海平,海上明月共潮生。\newline
            滟滟随波千万里,何处春江无月明?\newline
            江流宛转绕芳甸,月照花林皆似霰。\newline
            空里流霜不觉飞,汀上白沙看不见。\newline
            江天一色无纤尘,皎皎空中孤月轮。\newline
            江畔何人初见月,江月何年初照人?\newline
            人生代代无穷已,江月年年祇相似。\newline
            不知江月待何人?但见长江送流水。\newline
            白云一片去悠悠,青枫浦上不胜愁。\newline
            谁家今夜扁舟子,何处相思明月楼?\newline
            可怜楼上月徘徊,应照离人妆镜台。\newline
            玉户帘中卷不去,捣衣砧上拂还来。\newline
            此时相望不相闻,愿逐月华流照君。\newline
            鸿雁长飞光不度,鱼龙潜跃水成文。\newline
            昨夜闲潭梦落花,可怜春半不还家。\newline
            江水流春去欲尽,江潭落月复西斜。\newline
            斜月沉沉藏海雾,碣石潇湘无限路。\newline
            不知乘月几人归,落月摇情满江树。\newline
        \end{center}
    \subsection{繁体中文版 Traditional Chinese}
        \begin{center}
            春江潮水連海平,海上明月共潮生。\newline
            灩灩隨波千萬裏,何處春江無月明?\newline
            江流宛轉繞芳甸,月照花林皆似霰。\newline
            空裏流霜不覺飛,汀上白沙看不見。\newline
            江天壹色無纖塵,皎皎空中孤月輪。\newline
            江畔何人初見月,江月何年初照人?\newline
            人生代代無窮已,江月年年祇相似。\newline
            不知江月待何人?但見長江送流水。\newline
            白雲壹片去悠悠,青楓浦上不勝愁。\newline
            誰家今夜扁舟子,何處相思明月樓?\newline
            可憐樓上月徘徊,應照離人妝鏡臺。\newline
            玉戶簾中卷不去,搗衣砧上拂還來。\newline
            此時相望不相聞,願逐月華流照君。\newline
            鴻雁長飛光不度,魚龍潛躍水成文。\newline
            昨夜閑潭夢落花,可憐春半不還家。\newline
            江水流春去欲盡,江潭落月復西斜。\newline
            斜月沈沈藏海霧,碣石瀟湘無限路。\newline
            不知乘月幾人歸,落月搖情滿江樹。\newline
        \end{center}
\newpage
\section{注释 Note}
    \subsection{脚注 Footnote}
    \begin{center}
        春江潮水连海平,海上明月共潮生。\newline
        滟滟\footnote{滟滟:波光荡漾的样子。}随波千万里,何处春江无月明?\newline
        江流宛转绕芳甸\footnote{芳甸: 开满花草的郊野。甸,郊外之地。},
        月照花林皆似霰\footnote{霰: 天空中降落的白色不透明的小冰粒。此处形容月光下春花晶莹洁白。}。\newline
        空里流霜不觉飞,汀上白沙看不见。\newline
        江天一色无纤尘,皎皎空中孤月轮。\newline
        江畔何人初见月,江月何年初照人?\newline
        人生代代无穷已,江月年年祇相似。\newline
        不知江月待何人?但见长江送流水。\newline
        白云一片去悠悠,青枫浦上不胜愁。\newline
        谁家今夜扁舟子,何处相思明月楼?\newline
        可怜楼上月徘徊,应照离人妆镜台。\newline
        玉户帘中卷不去,捣衣砧上拂还来。\newline
        此时相望不相闻,愿逐月华流照君。\newline
        鸿雁长飞光不度,鱼龙潜跃水成文。\newline
        昨夜闲潭梦落花,可怜春半不还家。\newline
        江水流春去欲尽,江潭落月复西斜。\newline
        斜月沉沉藏海雾,碣石潇湘无限路。\newline
        不知乘月几人归,落月摇情满江树。\newline
    \end{center}

    \newpage
    \subsection{参考文献 Ref}
    \subsubsection{使用方法 Method of Use}
        If wants to use bib file, one should notice that the package needed is bibtex, not biblatex.
        For choosing compile engine, there are three recipes:
        \begin{itemize}
            \item \XeLaTeX × 2
            \item \XeLaTeX $\rightarrow$ \BibTeX $\rightarrow$ \XeLaTeX × 2
            \item \XeLaTeX $\rightarrow$ makeglossaries $\rightarrow$ \BibTeX $\rightarrow$ \XeLaTeX × 2
        \end{itemize}
        Without changing bib file, one could run \XeLaTeX twice to save time.
    \subsubsection{示例 Example}
        祥子的手哆嗦得更厉害了,揣起保单,拉起车,几乎要哭出来。
        拉到个僻静地方,细细端详自己的车,\cite{laoshe}在漆板上试着照照自己的脸!
        越看越可爱,就是那不尽合自己的理想的地方也都可以原谅了,因为已经是自己的车了。
        把车看得似乎暂时可以休息会儿了,他坐在了水簸箕的新脚垫儿上,看着车把上的发亮的黄铜喇叭。
        他忽然想起来,今年是二十二岁\cite{einstein}。因为父母死得早,他忘了生日是在哪一天。
        自从到城里来,他没过一次生日。
        好吧,今天买上了新车,就算是生日\cite{knuthwebsite}吧,人的也是车的,好记,而且车既是自己的心血,简直没什么不可以把人与车算在一块的地方。

\newpage
\bibliographystyle{unsrt}
\bibliography{simple_CHN_article_bib}
\printglossary[title={词汇表 Notation}, nonumberlist]
\glsaddall[] % show all gls without mentioning them
\end{document}